\documentclass[secnumarabic,balancelastpage,amsmath,amssymb]{article}
%---------------------------------------ponemos los paquetes.--------------------------------------------%

\usepackage{amsmath}
\usepackage{amsfonts}
\usepackage{graphicx}      % tools for importing graphics
\usepackage[utf8]{inputenc} %poner acentos
\usepackage[spanish]{babel} %titulos y descripciones en español
\usepackage[margin=2.0cm]{geometry} %modificar margenes del documento
%cosas raras
\usepackage{bm}            % special bold-math package. usge: \bm{mathsymbol}
\usepackage[colorlinks=false]{hyperref}
\usepackage{amssymb}
\usepackage{lscape}
\usepackage{titlesec} % Allows customization of titles
\renewcommand\thesection{\Roman{section}} % Roman numerals for the sections
\titleformat{\section}[block]{\large\scshape\centering}{\thesection.}{1em}{} % Change the look of the section titles
\titleformat{\subsection}[block]{\large}{\thesubsection.}{1em}{} % Change the look of the section titles
\usepackage{html,makeidx,epsf}


%---------------------------------------inciamos el documento.--------------------------------------------%

\begin{document}


\title{\vspace{-15mm}\textbf{Calculo Diferencial e Integral III}}
\author{\\ \emph{Universidad Nacional Autónoma de México }}

\section{Proposición 8.}
Sea A y B subconjuntos cerrados de $\mathbb {R}^{n}$, entonces: \\

\noindent b) $A^{\mathrm{o}} \cap B^{\mathrm{o}} =  (A \cap B)^{\mathrm{o}}$ 

\quad 1.- $A^{\mathrm{o}} \cap B^{\mathrm{o}} \subseteq (A \cap B)^{\mathrm{o}} $


\begin{equation}
\begin{aligned}
x \in (A^{\mathrm{o}} \cap B^{\mathrm{o}}) &\Rightarrow   (x \in A^{\mathrm{o}}) \land  (x \in B^{\mathrm{o}}) \Rightarrow (\exists B_r(x) \subset A) \land (\exists B_r(x) \subset B) \Rightarrow \exists B_r(x) \subset (A\cap B) \\
& \Rightarrow x \in (A \cap B)^{\mathrm{o}} \\
& \therefore A^{\mathrm{o}} \cap B^{\mathrm{o}} \subseteq (A \cap B)^{\mathrm{o}}
\end{aligned}
\end{equation}


\quad 2.- $ (A \cap B)^{\mathrm{o}} \subseteq A^{\mathrm{o}} \cap B^{\mathrm{o}} $

\begin{equation}
\begin{aligned}
x \in (A \cap B)^{\mathrm{o}} & \Rightarrow \exists B_r(x) \subset (A \cap B) \Rightarrow (\exists B_r(x) \subset A) \land (\exists B_r(x) \subset B) \Rightarrow (x \in A^{\mathrm{o}})\land(x \in B^{\mathrm{o}}) \\
& \Rightarrow x\in (A \cap B)^{\mathrm{o}} \\
& \therefore  (A \cap B)^{\mathrm{o}} \subseteq A^{\mathrm{o}} \cap B^{\mathrm{o}}
\end{aligned}
\end{equation}

$\therefore  A^{\mathrm{o}} \cap B^{\mathrm{o}} =  (A \cap B)^{\mathrm{o}}$ 

\noindent c) $\overline{A} \cup \overline{B} = \overline{A \cup B}$ 

\quad 1.- $\overline{A} \cup \overline{B} \subseteq \overline{A \cup B}$
\begin{equation}
\begin{aligned}
x \in  (\overline{A} \cup \overline{B} ) &\Rightarrow (x \in \overline{A}) \lor (x \in \overline{B}) \Rightarrow (\forall B_r(x), B_r(x) \cap A \neq \emptyset) \lor (\forall B_r(x), B_r(x) \cap B \neq \emptyset) \\
& \Rightarrow (\forall B_r(x), B_r(x) \cap (A \cup B) \neq \emptyset) \Rightarrow x \in (\overline{A \cup B}) \\
& \therefore \overline{A} \cup \overline{B} \subseteq \overline{A \cup B}
\end{aligned}
\end{equation}

\quad 2.- $\overline{A \cup B}  \subseteq \overline{A} \cup \overline{B} $
\begin{equation}
\begin{aligned}
x \in (\overline{A \cup B}) & \Rightarrow \forall B_r(x), B_r(x) \cap (A \cup B) \neq \emptyset \Rightarrow  (\forall B_r(x), B_r(x) \cap A \neq \emptyset)  \cup (\forall B_r(x), B_r(x) \cap B \neq \emptyset) \\
& \Rightarrow(\forall B_r(x), B_r(x) \cap A \neq \emptyset) \lor (\forall B_r(x), B_r(x) \cap B \neq \emptyset) \\
& \Rightarrow (x \in \overline{A}) \lor (x \in \overline{B}) \Rightarrow x \in (\overline{A} \cup \overline{B}) \\
& \therefore \overline{A \cup B}  \subseteq \overline{A} \cup \overline{B}
\end{aligned}
\end{equation}

$\therefore \overline{A} \cup \overline{B} = \overline{A \cup B}$


$\overline{A \cap B}\subset \overline{A} \cap \overline{B}$ \\

d) Por la proposición 3 inciso b) sabemos que 

$A\subset\overline{A}$ y $B\subset\overline{B}$ por lo que $A\cap B\subset\overline{A} \cap \overline{B}$ y dado que $\overline{A} \cap \overline{B}$ es cerrado $\Rightarrow$ $\overline{A \cap B}\subset \overline{A} \cap \overline{B}$.
Si consideramos $A=(0,1)$ y $B=(1,2)$ $\Rightarrow$ $\overline{A \cap B}=\emptyset$ pero $\overline{A}=[0,1]$ y $\overline{B}=[1,2]$ $\Rightarrow$ $\overline{A} \cap \overline{B}={1}$. 


\end{document}

