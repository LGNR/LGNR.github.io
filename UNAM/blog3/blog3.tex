\documentclass[secnumarabic,balancelastpage,amsmath,amssymb]{article}
%---------------------------------------ponemos los paquetes.--------------------------------------------%


\usepackage{graphicx}      % tools for importing graphics
\usepackage[utf8]{inputenc} %poner acentos
\usepackage[spanish]{babel} %titulos y descripciones en español
\usepackage[margin=2.0cm]{geometry} %modificar margenes del documento
%cosas raras
\usepackage{bm}            % special bold-math package. usge: \bm{mathsymbol}
\usepackage[colorlinks=false]{hyperref}
\usepackage{amssymb}
\usepackage{lscape}
\usepackage{titlesec} % Allows customization of titles
\renewcommand\thesection{\Roman{section}} % Roman numerals for the sections
\titleformat{\section}[block]{\large\scshape\centering}{\thesection.}{1em}{} % Change the look of the section titles
\titleformat{\subsection}[block]{\large}{\thesubsection.}{1em}{} % Change the look of the section titles



%---------------------------------------inciamos el documento.--------------------------------------------%

\begin{document}


\title{\vspace{-15mm}\textbf{Calculo Diferencial e Integral III}}
\author{\\ \emph{Universidad Nacional Autónoma de México }}

\section{Proposición 2.}
\subsection{b) Si A y B son subconjuntos cerrados en $\mathbb {R}^{n}$, entonces $A\cup B$ y $A\cap B$ también lo son.}
Como $A$ y $B$ son cerrados esto implica que $A^{c}$ y $B^{c}$ son abiertos. Si utilizamos las leyes de De Morgan
$(A\cup B)^{c}=A^{c}\cap B^{c}$ y considerando que la intersección de dos conjuntos abierto es un conjunto abierto 
$\Rightarrow$ $(A\cup B)^{c}$ es abierto  $\therefore$   $A\cup B$ es cerrado.   

Por otro lado, dado que $(A\cap B)^{c}=A^{c}\cup B^{c}$ y sabemos que la unión de dos conjuntos abiertos es un conjunto abierto $\Rightarrow$ $(A\cup B)^{c}$ es abierto  $\therefore$  $A\cap B$ es cerrado.   
\end{document}

